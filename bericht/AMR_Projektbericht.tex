\documentclass{scrartcl}% siehe <http://www.komascript.de>
\usepackage{selinput}% Eingabecodierung automatisch ermitteln …
\usepackage{hyperref}
\SelectInputMappings{% … siehe <http://ctan.org/pkg/selinput>
  adieresis={ä},
  germandbls={ß},
}
\usepackage{graphicx}
\usepackage[ngerman]{babel}% Das Beispieldokument ist in Deutsch,
                % daher wird mit Hilfe des babel-Pakets
                % über Option ngerman auf deutsche Begriffe
                % und gleichzeitig Trennmuster nach den
                % aktuellen Rechtschreiberegeln umgeschaltet.
                % Alternativen und weitere Sprachen sind
                % verfügbar (siehe <http://ctan.org/pkg/babel>).
                
\setlength{\parskip}{0.2cm}  % 2mm Abstand zwischen zwei Absätzen
\setlength{\parindent}{0mm}  % Absätze nicht einziehen
\usepackage[backend=biber,
isbn=false,                     % ISBN nicht anzeigen, gleiches geht mit nahezu allen anderen Feldern
sortlocale=de_DE,               % Sortierung der Einträge für Deutsch
%sortlocale=en_US,              % Sortierung der Einträge für Englisch
autocite=inline,                % regelt Aussehen für \autocite (inline=\parancite)
hyperref=true,                  % Hyperlinks für Ziate
style=ieee                     % Zitate als Zahlen [1]
%style=alphabetic               % Zitate als Kürzel und Jahr [Ein05]
%style=authoryear                % Zitate Author und Jahr [Einstein (1905)]
]{biblatex} 
\addbibresource{literatur.bib}
\setlength{\bibitemsep}{1em}     % Abstand zwischen den Literaturangaben
\setlength{\bibhang}{2em}        % Einzug nach jeweils erster Zeile

% Trennung von URLs im Literaturverzeichnis (große Werte [> 10000] verhindern die Trennung)
\defcounter{biburlnumpenalty}{10} % Strafe für Trennung in URL nach Zahl
\defcounter{biburlucpenalty}{500}  % Strafe für Trennung in URL nach Großbuchstaben
\defcounter{biburllcpenalty}{500}  % Strafe für Trennung in URL nach Kleinbuchstaben

\begin{document}
% ----------------------------------------------------------------------------
% Titel (erst nach \begin{document}, damit babel bereits voll aktiv ist:
\titlehead{AMR}% optional
\subject{Projektbericht AMR}% optional
\title{Kartierung der WLAN Signalstärke}% obligatorisch
%\subtitle{Untertitel}% optional
\author{Florian Zirker, Lukas Arnecke}% obligatorisch
%\date{z.\,B. der Abgabetermin}% sinnvoll
\publishers{Prof. Dr. Thomas Ihme}% optional
\maketitle% verwendet die zuvor gemachte Angaben zur Gestaltung eines Titels
% ----------------------------------------------------------------------------
% Inhaltsverzeichnis:
\tableofcontents
% ----------------------------------------------------------------------------
% Gliederung und Text:

\section{Einleitung}

\section{Grundlagen}
\subsection{Pioneer 3-DX Roboter}
Der mobile Forschungsroboter Pioneer 3-DX ist eine Roboterplattform, die für fast alle Anwendungen modifizierbar ist. Er besteht im Grunde nur aus einem Motor, Reifen und einer Plattform, durch die benötigte Module angebracht werden können \cite{pioneer}. Die Steuerungseinheit ist so gebaut, dass sie über das \textit{Robot Operating System} (ROS) angesprochen werden kann.

\begin{figure}
	\centering
	\includegraphics{bilder/pioneerRoh.jpg}
	\caption{Pioneed 3-DX Roboter ohne Aufbauten}
	\label{pioneerRoh}
\end{figure}

Auf der Plattform ist ein Laser, ein Wlan-Modul und eine Rechner aufgebaut, mit deren Hilfe die Kartierung durchgeführt wird. Auf dem Rechner läuft Ubuntu 18.04 LTS, auf welchem wiederum ROS in der Version Melodic installiert ist.
\subsection{Robot Operating System}
\subsection{Kartierung}

\subsection{Wireless Local Area Network}
In dieser Sektion wird folgendes erklärt:
\begin{itemize}
	\item WLAN (Funkmodi, Standards, )
	\item Kanäle 2,4 GHz und 5GHz
	\item Messungen
	\item RSSI-Wert
	\item ...
\end{itemize}
% Zuständigkeit: Flo


\section{Problemstellung}
% Eventuell können wir hier noch weiter unterscheiden: Inbetriebnahme, Kartierung und WLAN-Messung. Bin mir nicht sicher. FZ

\section{Realisierung}
\subsection{Genutzte Tools}
Für das Projekt wurde der Rechner auf dem Roboter komplett neu aufgesetzt. Für ein möglichst modernes System wurde Ubuntu 18.04 LTS installiert, wofür dann ROS in der Version Melodic vorgesehen ist und aus diesem Grund installiert wurde.

Desweiteren 


\subsection{Architektur}
Wir setzen auf ROS. Was wiederum auf dem Betriebsystem sitzt. Wir nutzen das Messaging-System (Publish-Subscribe). Wir nutzen p2os was wiederum auf ??? nutzt. Wir schreiben unseren eigenen Knoten.

\subsection{Aufgetretene Probleme}
% Wollen wir  wirklich in einem Kapitel erklären was für Probleme wir hatten? Ich denke es macht mehr sinn die einzellnen Schritte genauer zu erklären und dann an der entsprechenden Stelle zu erklären was schwierig war. (Siehe folgende Kapitel)
ROS Melodic ist zwar veröffentlicht und für Ubuntu 18.04 vorgesehen, jedoch sind noch nicht alle Funktionen der Vorgängerversionen implementiert. Das führte dazu, dass manche Aufrufe der Launchscripte manuell umgeschrieben und ROS neu kompiliert werden musste.

Auch war das Extrahieren der Signalstärken erwies sich als Problem.
%Sollte Flo möglichst ausfüllen

\subsection{Inbetriebnahme und Installation}
% Alles veraltet. Vorhergehende Gruppe hat Docker ansazt gewält, nicht mehr lauffähig. Neues Betriebsystem. Neues ROS. Damit einige Probleme: Selbstcompilieren einiger Pakete nötig. 

\subsection{Simultaneous Localization and Mapping}

\subsection{Aufzeichnung der Fahrt}

\subsection{Kartenerstellung aus aufgezeichneter Fahrt}

\subsection{WLAN-Messung}
% Zuständig:Flo

\section{Ergebnisse}
%Was ist raus gekommen. Neue Erkenntnisse, Lösungen, Unsere zwei Karten.

\section{Zusammenfassung}


\section{Ausblick}
Die Datenerhebung funktioniert. Der nächste Schritt wäre, mit dem Roboter strukturiert ein Gebiet zu kartieren und eine Flächendeckende Abdeckung zu erstellen. Auch würde eine Lösung, mit der eine Heatmap in RVIZ oder direkt danach überlagert eingeblendet werden kann, die Nachbearbeitung deutlich vereinfachen.



% Literaturverzeichnis erzeugen
\begin{flushleft}
	\printbibliography
\end{flushleft}


\end{document}
